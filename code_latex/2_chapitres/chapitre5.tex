\chapter{Analyse de Sensibilité des Indicateurs S2 aux Caractéristiques du Passif}

Le but de ce chapitre est de tester la robustesse de la méthode d'agrégation selectionnée au chapitre 4. Pour ce faire, des chocs (réglementaires + autres) seront appliqués sur des portefeuilles de test générés via le générateur présenté au chapitre 3. L'objectif est d'analyser l'impact de ces chocs sur les indicateurs S2 (BE, SCR) en comparant les résultats obtenus sur les portefeuilles granulaires (sans agrégation) et les portefeuilles agrégés (avec la méthode d'agrégation retenue). Cette analyse permettra de vérifier que la méthode d'agrégation choisie préserve la sensibilité des indicateurs aux variations des caractéristiques du passif, assurant ainsi la fiabilité des analyses de risque et de solvabilité dans un contexte d'optimisation computationnelle.
\section{Définition des Scénarios de Sensibilité}
    \subsection{Création des portefeuilles de test via le générateur}
    % Votre texte ici...
    \subsection{Description des chocs sur les variables clés (âge, montant de la PM, etc.)}
    % Votre texte ici...
    \subsection{Scénario d'intégration d'un nouveau produit dans le portefeuille}
    % Votre texte ici...

\section{Analyse de l'Impact de l'Agrégation sur la Mesure des Chocs}
    \subsection{Comparaison des indicateurs S2 sur portefeuilles choqués granulaires et agrégés}
    % Votre texte ici...
    \subsection{Analyse de la fidélité de la méthode d'agrégation à retranscrire la sensibilité}
    % Votre texte ici...

\section{Interprétation des Résultats et Validation de l'Approche}
    \subsection{Validation de la performance de la chaîne de modélisation (Générateur - Agrégation - Modèle ALM)}
    % Votre texte ici...
    \subsection{Enseignements sur la sensibilité des portefeuilles aux modifications de caractéristiques du passif}
    % Votre texte ici...

% \chapter{Résultats et Analyse des Sensibilités}

% % Introduction du chapitre : Expliquer que ce chapitre présente les résultats
% % de l'application du protocole défini au chapitre 4. C'est le "quoi" de l'analyse.

% \section{Application de la Méthode d'Agrégation sur les Portefeuilles de Test}
% % Appliquer la méthode choisie en 4.3 sur les portefeuilles définis en 4.2.

% \subsection{Analyse des Portefeuilles Agrégés}
% % Analyser brièvement les portefeuilles agrégés.
% % Question clé à répondre ici : Les chocs et modifications sont-ils toujours
% % visibles et bien représentés après l'agrégation ?

% \section{Analyse des Sensibilités sur les Indicateurs Solvabilité 2}
% % C'est le cœur de vos résultats.
% % Présenter les résultats issus du modèle ALM pour chaque scénario de sensibilité.
% % Comparer les indicateurs (BE, SCR) entre le portefeuille de base et les portefeuilles modifiés.

% \section{Interprétation des Résultats et Validation de l'Approche}
% % Prendre du recul sur les résultats chiffrés.

% \subsection{Évaluation de la performance du générateur}
% % Dans quelle mesure les résultats confirment-ils que le générateur produit
% % des portefeuilles réactifs et cohérents ?

% \subsection{Synthèse des impacts observés}
% % Quelles sont les différences clés observées ? Quels sont les enseignements
% % principaux des tests de sensibilité ?
